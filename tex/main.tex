\documentclass[a4paper,12pt]{scrartcl}

\usepackage[utf8]{inputenc}
\usepackage[english]{babel}
\usepackage[T1]{fontenc}
\usepackage{fancyhdr}
\usepackage{amsmath}
\usepackage{mathtools}
\usepackage{latexsym}
\usepackage{tensor}
\usepackage{graphicx}
\usepackage{tikz}
\usepackage{hyperref}
\usepackage{booktabs}
\usepackage{textcomp}

\allowdisplaybreaks

\title{Statistical Physics}
\subtitle{Homework, Sheet 7}
\author{Daniel Scheiermann --- 3227680 \\ Felix Springer --- 10002537 \\ Hinnerk --- 10002310}
\date{\today}

\begin{document}

\maketitle

If you have trouble understanding the program we would be happy to explain it, you can contact us via \texttt{stud.IP}.

\section{The Ising model [H13]}
We are using Python 3 for our solution. The mainly used modules are ''matplotlib'' and ''numpy''.
In order to understand what our program actually does it's best to start by looking in \textit{main.py} since most of the work is done here.

\begin{align}
	\tau_c \approx 10 \notag
\end{align}

\section{main.py}
The whole file pretty much consists of only one \textit{class} called \textit{grid}. The idea is to have all defining information of the grid in one place and beeing able to change it effortlessly. 

\subsection{\_\_init\_\_()}
This function is called when initalizing an object of this class. The function takes the dimensions of the grid as parameters. \\
The first grid data is created randomly using the numpy module. Also there are some variables set to their start value.

\subsection{print\_settings()}
This function prints the dimensions of the grid and also the current values of each element.

\subsection{update()}
Here the metroplis algorithm comes to use and calculates new values for the grid.

\subsection{make\_plot()}
By using \textit{matplotlib.pyplot} and \textit{matplotlib.animation} a plot of the grid is created. Or a little more precie: We used \textit{matshow()} in order to visualize our data for one single grid but also used \textit{FuncAnimation} so that we could create an actual animation of changing data. \\
Also a Slider for the temperature and the animation speed is created.





\end{document}
